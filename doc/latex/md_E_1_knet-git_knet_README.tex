\subsection*{}

{\bfseries knet} is cross-\/platform multi-\/threading light-\/weight T\+C\+P transport library, for now, knet support {\bfseries I\+O\+C\+P}, {\bfseries select}, {\bfseries epoll} model on {\bfseries Windows} and {\bfseries Cent\+O\+S}. it will testing under more Linux and Unix O\+S version, more model will be implemented like\+: kqueue(kevent) etc. {\bfseries knet} is not a event loop like {\bfseries libevent}, {\bfseries knet} focus on transport only.

\subsubsection*{loop}

\subsection*{}

{\ttfamily loop\+\_\+t} is the wrapper of different implemented model, it\textquotesingle{}s easy to use {\ttfamily loop\+\_\+t} to build a loop. \begin{DoxyVerb}#include "knet.h"
loop_t* loop = loop_create();
\end{DoxyVerb}


For now, a loop is ready for run, we create a {\ttfamily channel\+\_\+ref\+\_\+t} as acceptor. \begin{DoxyVerb}channel_ref_t* channel = loop_create_channel(loop, 0, 1024);
channel_ref_accept(channel, "127.0.0.1", 80);
\end{DoxyVerb}


We built a acceptor channel with infinite send chain size and maximal read buffer size 1024, after that run endless loop. \begin{DoxyVerb}loop_run(loop);
loop_destroy(loop);
\end{DoxyVerb}


The program just holds your screen, we should add some code for new client coming. \begin{DoxyVerb}void acceptor_cb(channel_ref_t* channel, channel_cb_event_e e) {
    if (e & channel_cb_event_accept) {
        /* TODO do job here */
    }
}
\end{DoxyVerb}


The whole story\+: \begin{DoxyVerb}#include "knet.h"

void acceptor_cb(channel_ref_t* channel, channel_cb_event_e e) {
    if (e & channel_cb_event_accept) { /* the new client coming */
        /* TODO do job here */
    }
}

int main() {
    loop_t* loop = loop_create();
    channel_ref_t* channel = loop_create_channel(loop, 0, 1024);
    channel_ref_accept(channel, "127.0.0.1", 80);
    channel_ref_set_cb(channel, acceptor_cb);
    loop_run(loop);
    loop_destroy(loop);
    return 0;
}
\end{DoxyVerb}


For more detail, see {\ttfamily \hyperlink{basic_8c}{examples/basic.\+c}}

\paragraph*{\hyperlink{config_8h}{config.\+h}}

\subsection*{}

\begin{DoxyVerb}#if defined(WIN32)
    #define LOOP_IOCP 1    /* IOCP */
    #define LOOP_SELECT 0  /* select */
#else
    #define LOOP_EPOLL 1   /* epoll */
    #define LOOP_SELECT 0  /* select */
#endif /* defined(WIN32) */
\end{DoxyVerb}


In header file {\ttfamily \hyperlink{config_8h}{knet/config.\+h}}, change the macro value to tell compiler choose specific implement, I\+O\+C\+P on Windows, epoll on Linux by default, more implement will be appended in the future release. \begin{DoxyVerb}#define LOGGER_ON 0 /* the switch of internal logger */
#define LOGGER_MODE (logger_mode_file | logger_mode_console | logger_mode_flush | logger_mode_override) /* the mode of logger */
#define LOGGER_LEVEL logger_level_verbose /* the level of logger */
\end{DoxyVerb}


{\ttfamily L\+O\+G\+G\+E\+R\+\_\+\+O\+N} is the switch of {\bfseries knet} internal logger, the macro {\ttfamily L\+O\+G\+G\+E\+R\+\_\+\+M\+O\+D\+E} and {\ttfamily L\+O\+G\+G\+E\+R\+\_\+\+L\+E\+V\+E\+L} can change the mode and the level of logger. Internal logger may help developer find the problom A\+S\+A\+P, {\ttfamily L\+O\+G\+G\+E\+R\+\_\+\+O\+N} should be set to 0 in release version.

\subsubsection*{balancer}

\subsection*{}

{\ttfamily loop\+\_\+t} runs at the thread which calling {\ttfamily loop\+\_\+run} or {\ttfamily loop\+\_\+run\+\_\+once}, each {\ttfamily loop\+\_\+t} knows nothing about others. {\ttfamily loop\+\_\+balancer\+\_\+t} coordinates all attached {\ttfamily loop\+\_\+t}({\ttfamily loop\+\_\+balance\+\_\+attach}) and try to balance load for them.

{\ttfamily loop\+\_\+balancer\+\_\+t} conceal the details and sophisticated for developer, the process of balancing is thread-\/safety and highly efficent. {\bfseries knet} offers a simple thread A\+P\+I to run internally {\ttfamily loop\+\_\+run}, the interact among loops are also transparent for developers.

For more detail, see {\ttfamily \hyperlink{multi__loop_8c}{examples/multi\+\_\+loop.\+c}}

\subsubsection*{rpc}

\subsection*{}

{\bfseries knet} offers a low level {\bfseries R\+P\+C} object framework for developer. {\ttfamily krpc\+\_\+t} is the base for the next version(1.\+5.\+x), developers may develop handwriting code by {\ttfamily krpc\+\_\+t}, but its not a good way to maintain code for future development.

For more detail, see {\ttfamily \hyperlink{examples_2rpc_8c}{examples/rpc.\+c}}

\subsubsection*{rpc code generating}

\subsection*{}

{\bfseries krpc} is a command line tool to generate R\+P\+C code for developes. The generated code utilizes R\+P\+C object framework to marshal/unmarshal the R\+P\+C method parameters, and a simple singleton R\+P\+C entry class.

{\ttfamily krpc -\/d output-\/directory -\/f input\+\_\+rpc\+\_\+file -\/n rpc\+\_\+class\+\_\+name -\/l the\+\_\+language\+\_\+to\+\_\+generate}

After run the command, you\textquotesingle{}ll find two files in the output-\/directory, {\ttfamily rpc\+\_\+class\+\_\+name.\+h} and {\ttfamily rpc\+\_\+class\+\_\+name.\+cpp}. For now, {\ttfamily krpc} supports C++ only.

For more detail, see


\begin{DoxyItemize}
\item {\ttfamily krpc/examples/rpc\+\_\+sample.\+rpc}
\item {\ttfamily krpc/examples/rpc\+\_\+sample\+\_\+object.\+rpc}
\item {\ttfamily \hyperlink{rpc_8cpp}{krpc/examples/rpc.\+cpp}}
\item {\ttfamily \hyperlink{rpc__sample_8h}{krpc/examples/rpc\+\_\+sample.\+h}}
\item {\ttfamily \hyperlink{rpc__sample_8cpp}{krpc/examples/rpc\+\_\+sample.\+cpp}}
\end{DoxyItemize}

\subsubsection*{build}

\subsection*{}


\begin{DoxyEnumerate}
\item Windows see {\ttfamily knet/win-\/proj/knet-\/vs2010.\+sln}
\item Linux see {\ttfamily knet/\+C\+Make\+Lists.\+txt} {\ttfamily cmake C\+Make\+Lists.\+txt} {\ttfamily make}
\end{DoxyEnumerate}

The build result in folder {\ttfamily knet/bin}.

\subsubsection*{test}

see {\ttfamily knet/bin}.


\begin{DoxyEnumerate}
\item Windows {\ttfamily test\+\_\+client-\/vs2010.\+exe} {\ttfamily test\+\_\+server-\/vs2010.\+exe}
\item Linux {\ttfamily test\+\_\+client} {\ttfamily test\+\_\+server}
\item The command line of client
\begin{DoxyEnumerate}
\item {\ttfamily -\/ip} server listening I\+P
\item {\ttfamily -\/port} server listening port
\item {\ttfamily -\/n} connector count
\end{DoxyEnumerate}
\item the command line of server
\begin{DoxyEnumerate}
\item {\ttfamily -\/ip} server listening I\+P
\item {\ttfamily -\/port} server listening port
\item {\ttfamily -\/w} worker count 
\end{DoxyEnumerate}
\end{DoxyEnumerate}